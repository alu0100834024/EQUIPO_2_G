\section{Algoritmo XXX}
\label{Apendice1:XXX}

\begin{center}
\begin{footnotesize}
\begin{verbatim}
###################################################################################
# Fichero .py
###################################################################################
#
# AUTORES Nayra Kintan Díaz, Javier de León Morales, José Eduardo Lorenzo Pérez
#   
# FECHA  11 de mayo de 2014
#
# DESCRIPCION Programa Python método de Newton: f(x)=cos(x)
#
###################################################################################

#!/usr/bin/python
#! enconding: utf-8
import math
import time

startubuntu=time.time()
startcpu=time.clock()
def funcionDada(x):
      return math.cos(x) - x

def derivadaFuncionDada(x):
      return -math.sin(x) - 1

i = int(0);

x = float(input("Ingrese el valor de x: "))

tempo = 0;
while(x != tempo and i<100):
     tempo = x
     x = x- funcionDada(x)/derivadaFuncionDada(x)
     e = abs ((x-tempo)/x)

     print("x" + str(i) + "=" + str(x) + "error=" + str(e) + "\n")
     i=i+1

if(i==100):
    print("\n\nNo converge")

else:
    print("\n\nSolucion x: " + str(x))
    
finishubuntu=time.time()
finishcpu=time.clock()
timecpu=finishcpu-startcpu
timeubuntu=finishubuntu-startubuntu
print 'Tiempo Ubuntu:%2.10f \n Tiempo CPU: %2.10f' %(timeubuntu,timecpu)
\end{verbatim}
\end{footnotesize}
\end{center}

\section{Algoritmo YYY}
\label{Apendice1:YYY}

\begin{center}
\begin{footnotesize}
\begin{verbatim}
/###################################################################################
 # Fichero .h
 ###################################################################################
 #
 # AUTORES  Nayra Kintan Díaz, Javier de León Morales, José Eduardo Lorenzo Pérez
 #
 # FECHA    11 de mayo de 2014
 #
 # DESCRIPCION  Programa Python que imprime los datos del hardware y del software de la máquina
 #
 ##################################################################################
#!enconding:UTF-8
#!/usr/bin/python

import platform
import os


def SOFTinfo():
  softinfo=()
  softinfo={'Several':platform.uname(),'S.O':platform.platform(),
 'Pythons Version':platform.python_version(), 'Date':platform.python_build()}
  return softinfo

def CPUinfo():
  # infofile on Linux machines:
  infofile ='/proc/cpuinfo'
  cpuinfo = {}
  if os.path.isfile(infofile):
    f = open(infofile, 'r')
    for line in f:
      try:
	name, value = [w.strip() for w in line.split(':')]
      except:
	continue
      if name == 'model name':
	cpuinfo['CPU type'] = value
      elif name == 'cache size':
	cpuinfo['cache size'] = value
      elif name == 'cpu MHz':
	cpuinfo['CPU speed'] = value + ' Hz'
      elif name == 'vendor_id':
	cpuinfo['vendor ID'] = value
    f.close()
  return cpuinfo

if __name__ == '__main__':
  softinfo=SOFTinfo()
  for keys in softinfo.keys():
    print softinfo[keys]
\end{verbatim}
\end{footnotesize}
\end{center}
