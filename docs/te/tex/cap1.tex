
Por el presente trabajo pretendemos ampliar los conocimientos acerca de las series de potencias hasta llegar a la aplicación del Método de Newton a la función $f(x)=cos(x).$ \\
Para la realización de este trabajo nos basamos en el uso de \LaTeX{}, la programación en Python, el uso del procesador de texto Kate, la utilización de Beamer y en general el trabajo con Linux.\\
\LaTeX{} es un procesador de texto con un lenguaje de bajo nivel. Hemos aprendido a insertar gráficas para representar ecuaciones, fórmulas, etc.\\
Nos permite estructurar fácilmente el documento con capítulos, secciones, notas, bibliografía, índices analíticos, lo que nos ayuda en la realización de este trabajo.\\
Beamer es un tipo de documento que está diseñado para presentaciones que utilicen recursos de \LaTeX{}, requiere la compilación a través de PDF\LaTeX{}. \\
Este nos fue útil para introducir fórmulas matemáticas, gráficos, imágenes, etc. \\
En particular, podemos decir que realizamos un programa en Python para resolver el método de Newton aplicado a la función trigonométrica $cos(x)$.\\
Vamos a desarrollar los contenidos sobre las series de potencias y sus aplicaciones.

